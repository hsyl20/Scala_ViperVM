\chapter{Introduction}

ViperVM \marginnote{ViperVM initially stands for "Very High Performance Virtual
Machine"} aims to be  a framework that allows high-performance applications to
be easily written and executed on any many-core cluster composed of homogeneous or
heterogeneous nodes (many-core, GPGPU, Cell\ldots). It has been conceived as a
way for researchers of different communities to work together.

ViperVM uses a high-level approach to application programming -- application
programmers don't have to know anything about the execution environment and
architecture -- while striving to get as much performances as possible.

\begin{description}
  \item[Programming languages] Programs that want to use this framework need to
  use some of the proposed domain specific languages (DSL).  Researchers of the
  programming language (PL) community may propose different DSL for different
  domains.

  \item[Scheduling] Different kinds of schedulers can be used to schedule both
  data transfers and computations.  Different scheduling models are available
  and new one could be proposed.

  \item[Compiler] Programs written using high-level representations must be
  compiled to native codes that can be executed by different kind of
  architectures (CPU, GPU\ldots).  Different strategies and optimizations may be
  used.

  \item[Auto-tuning] Different auto-tuning strategies may be proposed in order
  to get the best codes from compilers for each architecture

  \item[Numerical analysis] Optimizations should not only be performance guided
  but they should take accuracy into account too.  Numerical analysis methods
  should be employed in order to let the framework only use safe strategies.

  \item[System] System researchers may improve the architecture model used by
  the framework and make it portable on new architectures.  They could also
  improve the way the framework interacts with underlying hardware and operating
  systems.

\end{description}
