\chapter{Basic Runtime Systems}

Memory management and kernel scheduling are hard to do when several memory nodes
are available as well as different processors of different kinds (heterogeneous
architectures). Application programmers shouldn't have to manage all of this.

\subsection{Meta Views}
A first abstraction is what we call \textbf{meta views}. A meta view is
basically a set of views of the same kind that may be present in different
memories. Kernel instances are now configured using meta views instead of usual
views. When a kernel is to be executed by a given processor, the runtime system
ensures for each meta view used as parameter that a view is available in the
memory the processor is attached to. It performs any necessary data transfer and
manages the memory coherency between views of a meta view.

\subsection{Meta Kernels}
Application programmers have to select processors that will execute kernels. To
relieve programmers from this tedious task, we use an abstraction similar to
meta views: \textbf{meta kernels}. Similarly to meta views, meta kernels are
sets of kernels that can be executed by different processors and perform the
same operation.

Meta kernels can be used with or without meta views. When usual views are used,
the runtime system only has to select a kernel in the meta kernel compatible
with one of the processors attached to memory node containing views. When meta
views are used with meta kernels, the runtime system has to select a processor,
a kernel and to prepare views in memory attached to chosen processor.
