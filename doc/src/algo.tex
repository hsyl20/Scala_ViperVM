\chapter{Scheduling algorithms}

\begin{itemize}
  \item First phase consists in typing every term. Type is known for each term in the
program graph.
  \item Execution by graph rewriting.
  \item Two levels: meta data and data.
  \item Meta data are what is usually called "dependent types". For instance, matrix
  \item Terms: App, Id and Var
  \item Initial variables have type, meta data and some instances
dimensions, etc.
  \item Id and Var terms are values
  \item Reduction of App
  \begin{itemize}
    \item Given the function name, we know which param meta data and data are
    required to compute result meta data
    \item Given the function name, we know which param meta data and data are
    required to check whether some rewriting rules should be used
    \item Given the function name, we get different function implementations
    \item Implementations can be filtered by: target processor, param
    representations
  \end{itemize}
\end{itemize}
