\chapter{Introduction}

This document describes the ViperVM framework\marginnote{ViperVM initially stands for "Very High Performance Virtual Machine"}.
This framework allows high-performance applications to be written and to be executed on many-core clusters composed of homogeneous or heterogeneous nodes (many-core, GPGPU, Cell\ldots).
In its current state, it is mostly a research framework that allows researchers from different areas to work together.
\begin{description}
  \item[Programming languages] Programs that want to use this framework need to use some of the proposed domain specific languages (DSL).
        Researchers of the programming language (PL) community may propose different DSL for different domains.
  \item[Scheduling] Different kinds of schedulers can be used to schedule both data transfers and computations.
        Different scheduling models are available and new one could be proposed.
  \item[Compiler] Programs written using high-level representations must be compiled to native codes that can be executed by different kind of architectures (CPU, GPU\ldots).
        Different strategies and optimizations may be used.
  \item[Auto-tuning] Different auto-tuning strategies may be proposed in order to get the best codes from compilers for each architecture
  \item[Numerical analysis] Optimizations should not only be performance guided but they should take accuracy into account too.
        Numerical analysis methods should be employed in order to let the framework only use safe strategies.
  \item[System] System researchers may improve the architecture model used by the framework and make it portable on new architectures.
        They could also improve the way the framework interacts with underlying hardware and operating systems.
\end{description}
