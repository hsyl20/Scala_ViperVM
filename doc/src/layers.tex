\chapter{Layers}

\section{Domain Specific Languages}
ViperVM is composed of several modules.
The most important ones for users are DSLs (Domain Specific Language).
They provide languages for different domains such as linear algebra, logic or graph theory.
You can mix domains such that for instance a data generated with a linear algebra algorithm can be used into a graph algorithm.

DSL codes are converted by DSL engines into task graphs.
Different kinds of tasks are distinguished:
\begin{itemize}
  \item Actions: tasks performing side-effects (printing in terminal, writing a file, etc.)
  \item Pure tasks: tasks that don't perform any side-effect
\end{itemize}

Generated tasks must be supported by the runtime system.

\section{Runtime System}
The runtime system executes tasks submitted by DSL engines.
Programs can customize the runtime system they use.

\subsection{Task-Graph Optimizers}
Task graph optimizers can perform the following optimizations:
\begin{itemize}
  \item Common sub-expression elimination
  \item Task fusion
  \item Dead-code elimination
  \item Indicate if tasks can be executed in-place
\end{itemize}

\subsection{Generic Code Producers}
The generic code producer generates a code that can be compiled for any target.
Generic code is still high-level code.

\subsection{Generic Code Optimizers}
Optimizes generic code.

\subsection{Generic Code Compilers}
Compiles generic code for different targets: OpenCL, x86, JVM, CUDA, etc.

\subsection{Schedulers}

\subsection{Backends}
The Runtime System supports different backends for different architectures: x86, JVM, CUDA, OpenCL\ldots
