\chapter{Platforms}

ViperVM is designed to run on any kind of platform. For instance, it can be
used on single-core, multi-core or many-core architectures, with or without
additional accelerators (GPU, CELL\ldots). It also manages clusters composed of
such architectures.

ViperVM models architectures using the following entities:

\begin{description}

  \item[Memory] This is the main part of most architectures. Memories contain
  data that can be used to perform computations.

  \item[Buffer] A buffer is a contiguous space in a memory.

  \item[BufferView] A buffer view is a contiguous or discontiguous memory space
  in a buffer.

  \item[Network] This is a physical network that links together some memories.

  \item[Link] A link is a directed point-to-point association between two
  memories using a network. It can be represented as a 3-tuple (Network,
  SourceMemory, DestinationMemory).

  \item[Processor] A processor is a compute unit that can execute tasks.
  Processors have one or more associated memories that can be read or write to
  by tasks.

  \item[Kernel] A kernel is a piece of code that can be executed by some kind of
  processor. Kernels can have parameters of different types: buffers or immediate
  values (integer, floating point numbers\ldots).

\end{description}

\section{Platform configuration}
ViperVM provides a Platform class that can be configured to use some backends.
There are backends for the JVM, for CUDA, for OpenCL, for native C code,
etc.\footnote{Currently these backends are not production ready or not even
available, but they should be the first to be provided} and more of them could
be provided in the future. Using these backends, Platforms objects are able to
retrieve information about available memories, processors and networks.

\section{Asynchronous operations}

