\chapter{Runtime System}

\section{Platform}

\subsection{Drivers}
Drivers manage computing devices, memory nodes and networks.


\subsection{Memory node}
A memory node contains data.
It may also contain compiled programs as some device require explicit management of those (CELL).

\begin{description}
  \item[allocate(size): Buffer] Allocate a buffer on this node
  \item[buffers] List of buffers on this memory node
\end{description}

\subsection{Network}
A network links different memory nodes together.
Some networks are in fact DMA controllers + bus, other use a computing unit to transfer data (NUMA), other are really NIC+network. 
Some networks can perform different transfers (up to a certain number) at the same time. Some can't.


\paragraph{Link}
A link is a virtual one-way channel between two memory nodes. Each link is associated to a single network.


\subsection{Processor}
A processor can execute programs. These programs may need to be stored in some specific memory (CELL, GPU kernels (hidden to programmer)).
Programs work on data. These data are stored in buffers on specific memory.


\section{Data Management}

\subsection{Buffers}
A buffer is a contiguous memory space in a memory node.
It cannot be used directly. 
Views are used to access buffer contents.

\subsection{Buffer Views}
A buffer view is a region of a buffer.
There are different kind of views: contiguous (1D), 2D, 3D\ldots
Views are used as sources and targets for data copy operations.

\subsection{Data}
A Data is a set of buffer views.
These buffer views are usually on different memory nodes.
It is possible to add or remove views from a Data.

There are different types of Data.
Each data type is backed by a buffer view type.
For instance, 2D Matrix data type is backed by 2D buffer views.

\subsection{Data Configuration}
A data configuration is a set of data.
Data configurations are used for kernel execution requirements:
all kernel parameters are stored in a data configuration and the runtime system must set up this data configuration on a memory node for the kernel to be executed.

Data configurations support constraints:
\begin{itemize}
  \item Memory nodes that shouldn't be used
  \item Memory nodes that may be used
  \item Memory node affinity rank (i.e. priorities)
\end{itemize}

\section{Code Management}

\subsection{Kernel}
A kernel is a short piece of code that can be executed by one device.
Kernels support different kinds of parameters such as Buffers and immediate values.

\subsection{Meta Kernels}
A meta kernel is a kernel that can be executed by different kinds of devices.
It may be be a single code that is compilable for different devices or a collection of kernels.

