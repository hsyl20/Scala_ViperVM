\chapter{Runtime System}

\section{Drivers}
Drivers manage computing devices, memory nodes and networks.


\section{Memory node}
A memory node contains data.
It may also contain compiled programs as some device require explicit management of those (CELL).

\begin{description}
  \item[allocate(size): Buffer] Allocate a buffer on this node
  \item[buffers] List of buffers on this memory node
\end{description}


\section{Buffer}
A contiguous memory space in a memory node

\begin{description}
  \item[free] Free this buffer
  \item[memoryNode] Associated memory node
\end{description}


\section{Network}
A network links different memory nodes together.
Some networks are in fact DMA controllers + bus, other use a computing unit to transfer data (NUMA), other are really NIC+network. 
Some networks can perform different transfers (up to a certain number) at the same time. Some can't.


\section{Link}
A link is a virtual one-way channel between two memory nodes. Each link is associated to a single network.


\section{Computing device}
A computing device can execute programs. These programs may need to be stored in some specific memory (CELL, GPU kernels (hidden to programmer)).
Programs work on data. These data are stored in buffers on specific memory.


