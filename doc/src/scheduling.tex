\chapter{Schedulers}

Schedulers are at the core of the runtime system.
They are responsible for task placement, task scheduling, data allocation/deallocation and data transfer scheduling.

\section{Functional Scheduler}
A functional scheduler supports functional kernels.
It allows applications to submit functional kernels with their parameters and to discard data that are no longer necessary.

\subsection{ActorFunctionalScheduler trait}
ActorFunctionalScheduler trait extends a functional scheduler to make it use the actor library to deal with asynchronous events.
It is based on the following messages:
\begin{itemize}
  \item TaskSubmitted(task): a task has been submitted by the application
  \item DataDiscarded(data*): some data have been discarded by the application
  \item TaskCompleted(task): a task has completed its execution
  \item DataTransferCompleted(data,mem): a data transfer has completed
\end{itemize}

It provides the following protected helper methods:
\begin{itemize}
  \item monitor(taskEvent): emit TaskCompleted when taskEvent completes.
  \item monitor(dataTransferEvent): emit DataTransferCompleted when dataTransferEvent completes.
  \item execute(task,proc,mem) to indicate that a task should be executed by a given processor on a given memory. This method emits TaskExecute(task,proc,mem).
\end{itemize}

\subsection{Data transfers}
There are different kinds of data transfers that can occur:
\begin{itemize}
  \item A data doesn't exist in a memory node: a data copy from another memory node is necessary.
  \item A data already exists in a memory node but a duplicate is necessary 
\end{itemize}

\subsection{Dependency trait}
When a task is submitted to the scheduler, the latter has to wait for task's input data to be produced before it can schedule it.
The Dependency trait splits submitted tasks in two categories:
\begin{itemize}
  \item InputReadyTasks: tasks for which input data have been produced
  \item InputWaitingTasks : tasks that still need some input data to be produced
\end{itemize}

When TaskSubmitted(task) is received, this trait selects the appropriate category for the task.
When DataAvailable(data) is received, it moves tasks from InputWaitingTasks to InputReadyTasks when applicable and it emits TaskDataReady(task) for each moved task.

\subsection{Garbage collector trait}
Data should be discarded as soon as they are not necessary anymore as it may avoid superfluous transfers and may allow in-place data modifications.
When DataDiscarded(data) is received, this trait sets a flag Discarded(data) indicating that the data has been discarded.
If Tasks set contains no task using this data, the data is freed as well as Discarded(data).
When TaskCompleted(task) is received, if Discarded(data) is set and Tasks set doesn't contain any task using this data, the data is freed as well as Discarded(data).


\subsection{Task queues trait}
This trait associates a priority queue taskQueue(proc) to each processor.
When TaskExecute(task,proc,mem) is received, TaskQueuePush(task,queue) is emitted.
When TaskQueuePop(task,queue) is received, it removes task from queue.
It emits TaskQueueEmpty(queue) when a task queue becomes empty.

\subsubsection{Data transfer queues trait}
This trait manages priority queues of data that need to be transferred on a memory.
It associates a priority queue transferQueue(mem) to each memory.

\subsection{Default functional scheduler}
DataSynced(d) indicates that data d is available on host mem.

DataOn(m) : data present in memory m
IsOn(d,m) : indicate that d is on m
FutureDataOn(m)  : data currently being transferred to m
Refs(d,m) : number of tasks in RunQueue depending on d in m
FutureRefs(d,m) : number of tasks in SubmittedQueue depending on d in m


For every T in SubmittedQueue:
let ds be T's data set.
For every d in ds, increase FutureRefs(d,m)

If SubmittedQueue is not empty, pick a task T in it.
Select a proc p and a memory m.
Let ds be the task data set.
For every d in ds that is not in DataOn(m) U FutureDataOn(m), try to allocate a buffer and schedule data transfer if necessary
If there is not enough space available, let the task in the SubmittedQueue and execute cleaning(m,T).
For every d in ds, increase Refs(d,m) and decrease FutureRefs(d,m)
Otherwise, remove the task from the SubmittedQueue and put it in the RunQueue.

When a transfer of d is completed in m.
Select tasks Ts in RunQueue that can be executed.
Execute them, remove them from RunQueue and put them in RunningQueue.
Decrease Refs(d,m)

When a task T is completed.
Remove it from RunningQueue.
Let ds be the task data set.
For every d in ds, decrease Refs(d,m)

Cleaning(m,T)
Let required = {d in T's data set, not IsOn(d,m)}
Let present = {d in T's data set, IsOn(d,m)}
Let size be the size of T's data set intersection DataOn(m).
Let ds = {d in DataOn(m), DataSynced(d) is true}
Free data from ds in m up to size.
Let size2 be the remaining required memory size.
If size2 is zero, continue submission of T, otherwise postpone it and continue.
Schedule data transfers of data in DataOn(m)/present to host such as at least size2 data is transferred.
Increase Refs(d,m) for data transferred.
Select those with smaller Refs(d,m) and smaller FutureRefs(d,m)


\section{Data affinity}
When a kernel is to be scheduled, the scheduler must set up the data configuration composed of the kernel parameters on a memory node.
To avoid superfluous transfers, memory nodes can be ranked depending on the amount of data they already contain.
Moreover, transfer speed may also be taken into account to improve the ranking.
If data have to be evicted for the data configuration to be set up, the rank should take this into account.

Let $D_i (0 \le i \le n)$ be the data required by the kernel.
Let $R_i$ be the $D_i$ accessed in read-only mode and $W_i$ the ones accessed in read-write mode.
Each $W_i$ must be either duplicated or the scheduler must ensure that the data is not needed anymore.

Basic algorithm:

\begin{lstlisting}[language=scala]
def affinities(config:DataConfiguration): Seq[MemoryNode, Int] = {
  val ms = memoryNodes.filterNot(config.excluded.contains(_))
  for (m <- ms) yield {

    /* Get unavailable data on the node
     * Available data include data being transferred (willContain)
     */
    val (avail,unavail) = config.data.map(d => (d, m.willContain(d))).split(_._2)

    /* Get required memory */
    val mallocsize = unavail.map(_.sizeOn(m)).sum
    
    /* Get data that need to be initialized */
    val toinit = unavail.filter(_.initialized)

    /* Get transfer size */
    val toinit_size = toinit.map(_.sizeOn(m)).sum


    /* Return rank */
    (m, rank)
  }
}
\end{lstlisting}
